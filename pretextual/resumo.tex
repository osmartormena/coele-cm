Resumo na língua vernácula, elaborado conforme a ABNT NBR 6028:2021. O resumo deve ressaltar sucintamente o conteúdo de um texto. O resumo deve ser composto por uma sequência de frases concisas em parágrafo único, sem enumeração de tópicos. Convém usar o verbo na terceira pessoa. Convém evitar: símbolos, contrações, reduções, entre outros, que não sejam de uso corrente; fórmulas, equações, diagramas, entre outros, que não sejam absolutamente necessários, e, quando seu emprego for imprescindível, defini-los na primeira vez que aparecerem. As palavras-chave devem figurar logo abaixo do resumo, antecedidas da expressão \enquote{Palavras"-chave}, seguida de dois"-pontos, separadas entre si por ponto e vírgula e finalizadas por ponto. Devem ser grafadas com as iniciais em letra minúscula, com exceção dos substantivos próprios e nomes científicos. Quanto à sua extensão, convém que os resumos tenham 150 a 500 palavras nos trabalhos acadêmicos.

\vspace{\onelineskip}
\noindent Palavras"-chave: LaTeX; ABNT; UTFPR; trabalho de conclusão de curso.
